\documentclass[a4paperpaper,]{article}
\usepackage{lmodern}
\usepackage{amssymb,amsmath}
\usepackage{ifxetex,ifluatex}
\usepackage{fixltx2e} % provides \textsubscript
\ifnum 0\ifxetex 1\fi\ifluatex 1\fi=0 % if pdftex
  \usepackage[T1]{fontenc}
  \usepackage[utf8]{inputenc}
\else % if luatex or xelatex
  \ifxetex
    \usepackage{mathspec}
  \else
    \usepackage{fontspec}
  \fi
  \defaultfontfeatures{Ligatures=TeX,Scale=MatchLowercase}
\fi
% use upquote if available, for straight quotes in verbatim environments
\IfFileExists{upquote.sty}{\usepackage{upquote}}{}
% use microtype if available
\IfFileExists{microtype.sty}{%
\usepackage{microtype}
\UseMicrotypeSet[protrusion]{basicmath} % disable protrusion for tt fonts
}{}
\usepackage[left=3.75cm, right=3.75cm, top=3cm, bottom=3cm]{geometry}
\usepackage{hyperref}
\PassOptionsToPackage{usenames,dvipsnames}{color} % color is loaded by hyperref
\hypersetup{unicode=true,
            pdftitle={Travaux Pratiques de Biométrie 3},
            pdfauthor={Benoît Simon-Bouhet},
            colorlinks=true,
            linkcolor=Maroon,
            citecolor=Blue,
            urlcolor=blue,
            breaklinks=true}
\urlstyle{same}  % don't use monospace font for urls
\usepackage{natbib}
\bibliographystyle{apalike}
\usepackage{color}
\usepackage{fancyvrb}
\newcommand{\VerbBar}{|}
\newcommand{\VERB}{\Verb[commandchars=\\\{\}]}
\DefineVerbatimEnvironment{Highlighting}{Verbatim}{commandchars=\\\{\}}
% Add ',fontsize=\small' for more characters per line
\usepackage{framed}
\definecolor{shadecolor}{RGB}{247,247,247}
\newenvironment{Shaded}{\begin{snugshade}}{\end{snugshade}}
\newcommand{\AlertTok}[1]{\textcolor[rgb]{0.75,0.01,0.01}{\textbf{\colorbox[rgb]{0.97,0.90,0.90}{#1}}}}
\newcommand{\AnnotationTok}[1]{\textcolor[rgb]{0.79,0.38,0.79}{#1}}
\newcommand{\AttributeTok}[1]{\textcolor[rgb]{0.00,0.34,0.68}{#1}}
\newcommand{\BaseNTok}[1]{\textcolor[rgb]{0.69,0.50,0.00}{#1}}
\newcommand{\BuiltInTok}[1]{\textcolor[rgb]{0.39,0.29,0.61}{\textbf{#1}}}
\newcommand{\CharTok}[1]{\textcolor[rgb]{0.57,0.30,0.62}{#1}}
\newcommand{\CommentTok}[1]{\textcolor[rgb]{0.54,0.53,0.53}{#1}}
\newcommand{\CommentVarTok}[1]{\textcolor[rgb]{0.00,0.58,1.00}{#1}}
\newcommand{\ConstantTok}[1]{\textcolor[rgb]{0.67,0.33,0.00}{#1}}
\newcommand{\ControlFlowTok}[1]{\textcolor[rgb]{0.12,0.11,0.11}{\textbf{#1}}}
\newcommand{\DataTypeTok}[1]{\textcolor[rgb]{0.00,0.34,0.68}{#1}}
\newcommand{\DecValTok}[1]{\textcolor[rgb]{0.69,0.50,0.00}{#1}}
\newcommand{\DocumentationTok}[1]{\textcolor[rgb]{0.38,0.47,0.50}{#1}}
\newcommand{\ErrorTok}[1]{\textcolor[rgb]{0.75,0.01,0.01}{\underline{#1}}}
\newcommand{\ExtensionTok}[1]{\textcolor[rgb]{0.00,0.58,1.00}{\textbf{#1}}}
\newcommand{\FloatTok}[1]{\textcolor[rgb]{0.69,0.50,0.00}{#1}}
\newcommand{\FunctionTok}[1]{\textcolor[rgb]{0.39,0.29,0.61}{#1}}
\newcommand{\ImportTok}[1]{\textcolor[rgb]{1.00,0.33,0.00}{#1}}
\newcommand{\InformationTok}[1]{\textcolor[rgb]{0.69,0.50,0.00}{#1}}
\newcommand{\KeywordTok}[1]{\textcolor[rgb]{0.12,0.11,0.11}{\textbf{#1}}}
\newcommand{\NormalTok}[1]{\textcolor[rgb]{0.12,0.11,0.11}{#1}}
\newcommand{\OperatorTok}[1]{\textcolor[rgb]{0.12,0.11,0.11}{#1}}
\newcommand{\OtherTok}[1]{\textcolor[rgb]{0.00,0.43,0.16}{#1}}
\newcommand{\PreprocessorTok}[1]{\textcolor[rgb]{0.00,0.43,0.16}{#1}}
\newcommand{\RegionMarkerTok}[1]{\textcolor[rgb]{0.00,0.34,0.68}{\colorbox[rgb]{0.88,0.91,0.97}{#1}}}
\newcommand{\SpecialCharTok}[1]{\textcolor[rgb]{0.24,0.68,0.91}{#1}}
\newcommand{\SpecialStringTok}[1]{\textcolor[rgb]{1.00,0.33,0.00}{#1}}
\newcommand{\StringTok}[1]{\textcolor[rgb]{0.75,0.01,0.01}{#1}}
\newcommand{\VariableTok}[1]{\textcolor[rgb]{0.00,0.34,0.68}{#1}}
\newcommand{\VerbatimStringTok}[1]{\textcolor[rgb]{0.75,0.01,0.01}{#1}}
\newcommand{\WarningTok}[1]{\textcolor[rgb]{0.75,0.01,0.01}{#1}}
\usepackage{longtable,booktabs}
\usepackage{graphicx,grffile}
\makeatletter
\def\maxwidth{\ifdim\Gin@nat@width>\linewidth\linewidth\else\Gin@nat@width\fi}
\def\maxheight{\ifdim\Gin@nat@height>\textheight\textheight\else\Gin@nat@height\fi}
\makeatother
% Scale images if necessary, so that they will not overflow the page
% margins by default, and it is still possible to overwrite the defaults
% using explicit options in \includegraphics[width, height, ...]{}
\setkeys{Gin}{width=\maxwidth,height=\maxheight,keepaspectratio}
\IfFileExists{parskip.sty}{%
\usepackage{parskip}
}{% else
\setlength{\parindent}{0pt}
\setlength{\parskip}{6pt plus 2pt minus 1pt}
}
\setlength{\emergencystretch}{3em}  % prevent overfull lines
\providecommand{\tightlist}{%
  \setlength{\itemsep}{0pt}\setlength{\parskip}{0pt}}
\setcounter{secnumdepth}{5}
% Redefines (sub)paragraphs to behave more like sections
\ifx\paragraph\undefined\else
\let\oldparagraph\paragraph
\renewcommand{\paragraph}[1]{\oldparagraph{#1}\mbox{}}
\fi
\ifx\subparagraph\undefined\else
\let\oldsubparagraph\subparagraph
\renewcommand{\subparagraph}[1]{\oldsubparagraph{#1}\mbox{}}
\fi

%%% Use protect on footnotes to avoid problems with footnotes in titles
\let\rmarkdownfootnote\footnote%
\def\footnote{\protect\rmarkdownfootnote}

%%% Change title format to be more compact
\usepackage{titling}

% Create subtitle command for use in maketitle
\newcommand{\subtitle}[1]{
  \posttitle{
    \begin{center}\large#1\end{center}
    }
}

\setlength{\droptitle}{-2em}

  \title{Travaux Pratiques de Biométrie 3}
    \pretitle{\vspace{\droptitle}\centering\huge}
  \posttitle{\par}
    \author{Benoît Simon-Bouhet}
    \preauthor{\centering\large\emph}
  \postauthor{\par}
      \predate{\centering\large\emph}
  \postdate{\par}
    \date{2019-02-26}

\usepackage[french]{babel}
\usepackage{mathspec}  % \usepackage{fontspec}
\usepackage{natbib}

\usepackage{setspace,booktabs,rotating,placeins,hvfloat,textcomp}

\usepackage{graphicx}
\usepackage{multicol}
\usepackage{enumitem}
\usepackage{longtable}

\setallmainfonts[Ligatures = TeX]{FuturaLT-Book}
\onehalfspace

\begin{document}
\maketitle

{
\hypersetup{linkcolor=black}
\setcounter{tocdepth}{2}
\tableofcontents
}
\hypertarget{preambule}{%
\section{Préambule}\label{preambule}}

Ce livre contient l'ensemble du matériel (contenus, exemples, exercices\ldots{}) nécessaire à la réalisation des travaux pratiques et TEA de biométrie 3. Ces travaux pratiques ont un seul objectif principal : vous permettre de mettre en œuvre, dans \texttt{RStudio} les méthodes statistiques découvertes en cours magistral et en TD de biométrie 2 (au semestre précédent) et en biométrie 3 depuis début janvier.

Je considère qu'à ce stade, vous devez être à l'aise dans RStudio pour effectuer les tâches suivantes :

\begin{enumerate}
\def\labelenumi{\arabic{enumi}.}
\tightlist
\item
  Importer des jeux de données dans RStudio
\item
  Manipuler des tableaux de données avec \texttt{tidyr} pour les mettre dans un format permettant les analyses statistiques et les représentations graphiques
\item
  Faire des graphiques exploratoires avec \texttt{ggplot2} pour visualiser des données
\item
  Filtrer des lignes, sélectionner des colonnes, trier, créer de nouvelles variables et calculer des résumés des données avec les fonction \texttt{filter()}, \texttt{select()}, \texttt{arrange()}, \texttt{mutate()}, \texttt{summarise()} et \texttt{group\_by()} du package \texttt{dplyr}
\item
  Utiliser le pipe \texttt{\%\textgreater{}\%} afin d'enchaîner plusieurs commandes
\item
  Créer des scripts parlants contenant des commandes et des commentaires utiles
\item
  Spécifier/modifier votre répertoire de travail
\item
  Installer des packages additionnels.
\end{enumerate}

Si vous pensez avoir besoin de rappels sur ces notions, je vous encourrage vivement à consulter \href{https://besibo.github.io/Biometrie2/}{le livre en ligne} dédié aux travaux pratiques de Biometrie 2 pour vous rafraîchir la mémoire.

L'organisation des TP et TEA de biométrie 3 sera la suivante :

\begin{itemize}
\tightlist
\item
  Séance 1 : 1h30 de TP suivie d'une séance de 1h30 de TEA. Rappels concernant les statistiques descriptives et les visualisations graphiques utiles pour déméler la complexité de certains jeux de données. Comparaisons (paramétriques et non paramétriques) de la moyenne de 2 populations.
\item
  Séance 2 : 1h30 de TP suivie d'une séance de 1h30 de TEA. Comparaisons (paramétriques et non paramétriques) la moyenne de plus de 2 populations : analyse de variance, hypothèses et conditions d'application.
\item
  Séance 3 : 1h30 de TP suivie d'une séance de 1h30 de TEA. Étude de la liaison entre 2 variables. Corrélation (paramétrique et non paramétrique) et régression linéaire. Tests d'hypothèses, estimation et conditions d'application.
\item
  Séance 4 : 1h30 de TP. Exercices d'application et corrections en guise de préparation pour l'examen.
\end{itemize}

\hypertarget{intro}{%
\section{Introduction}\label{intro}}

You can label chapter and section titles using \texttt{\{\#label\}} after them, e.g., we can reference Chapter \ref{intro}. If you do not manually label them, there will be automatic labels anyway, e.g., Chapter \ref{methods}.

Figures and tables with captions will be placed in \texttt{figure} and \texttt{table} environments, respectively.

\begin{Shaded}
\begin{Highlighting}[]
\KeywordTok{par}\NormalTok{(}\DataTypeTok{mar =} \KeywordTok{c}\NormalTok{(}\DecValTok{4}\NormalTok{, }\DecValTok{4}\NormalTok{, }\FloatTok{0.1}\NormalTok{, }\FloatTok{0.1}\NormalTok{))}
\KeywordTok{plot}\NormalTok{(pressure, }\DataTypeTok{type =} \StringTok{"b"}\NormalTok{, }\DataTypeTok{pch =} \DecValTok{19}\NormalTok{)}
\end{Highlighting}
\end{Shaded}

\begin{figure}[htpb]

{\centering \includegraphics[width=0.8\linewidth]{figure/nice-fig-1} 

}

\caption{Here is a nice figure!}\label{fig:nice-fig}
\end{figure}

Reference a figure by its code chunk label with the \texttt{fig:} prefix, e.g., see Figure \ref{fig:nice-fig}. Similarly, you can reference tables generated from \texttt{knitr::kable()}, e.g., see Table \ref{tab:nice-tab}.

\begin{Shaded}
\begin{Highlighting}[]
\NormalTok{knitr}\OperatorTok{::}\KeywordTok{kable}\NormalTok{(}
  \KeywordTok{head}\NormalTok{(iris, }\DecValTok{20}\NormalTok{), }\DataTypeTok{caption =} \StringTok{'Here is a nice table!'}\NormalTok{,}
  \DataTypeTok{booktabs =} \OtherTok{TRUE}
\NormalTok{)}
\end{Highlighting}
\end{Shaded}

\begin{table}[t]

\caption{\label{tab:nice-tab}Here is a nice table!}
\centering
\begin{tabular}{rrrrl}
\toprule
Sepal.Length & Sepal.Width & Petal.Length & Petal.Width & Species\\
\midrule
5.1 & 3.5 & 1.4 & 0.2 & setosa\\
4.9 & 3.0 & 1.4 & 0.2 & setosa\\
4.7 & 3.2 & 1.3 & 0.2 & setosa\\
4.6 & 3.1 & 1.5 & 0.2 & setosa\\
5.0 & 3.6 & 1.4 & 0.2 & setosa\\
\addlinespace
5.4 & 3.9 & 1.7 & 0.4 & setosa\\
4.6 & 3.4 & 1.4 & 0.3 & setosa\\
5.0 & 3.4 & 1.5 & 0.2 & setosa\\
4.4 & 2.9 & 1.4 & 0.2 & setosa\\
4.9 & 3.1 & 1.5 & 0.1 & setosa\\
\addlinespace
5.4 & 3.7 & 1.5 & 0.2 & setosa\\
4.8 & 3.4 & 1.6 & 0.2 & setosa\\
4.8 & 3.0 & 1.4 & 0.1 & setosa\\
4.3 & 3.0 & 1.1 & 0.1 & setosa\\
5.8 & 4.0 & 1.2 & 0.2 & setosa\\
\addlinespace
5.7 & 4.4 & 1.5 & 0.4 & setosa\\
5.4 & 3.9 & 1.3 & 0.4 & setosa\\
5.1 & 3.5 & 1.4 & 0.3 & setosa\\
5.7 & 3.8 & 1.7 & 0.3 & setosa\\
5.1 & 3.8 & 1.5 & 0.3 & setosa\\
\bottomrule
\end{tabular}
\end{table}

You can write citations, too. For example, we are using the \textbf{bookdown} package \citep{R-bookdown} in this sample book, which was built on top of R Markdown and \textbf{knitr} \citep{xie2015}.

\hypertarget{literature}{%
\section{Literature}\label{literature}}

Here is a review of existing methods.

\hypertarget{methods}{%
\section{Methods}\label{methods}}

We describe our methods in this chapter. And so much more\ldots{}

\hypertarget{applications}{%
\section{Applications}\label{applications}}

Some \emph{significant} applications are demonstrated in this chapter.

\hypertarget{example-one}{%
\subsection{Example one}\label{example-one}}

\hypertarget{example-two}{%
\subsection{Example two}\label{example-two}}

\hypertarget{final-words}{%
\section{Final Words}\label{final-words}}

We have finished a nice book.

\bibliography{book.bib,packages.bib}


\end{document}
